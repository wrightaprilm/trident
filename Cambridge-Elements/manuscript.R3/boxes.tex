\begin{boxedtext}{The likelihood the prior and the posterior} 
%for some reason you can't have a comma in the box titles
It can be confusing in the beginning to understand what the model likelihood, the prior, \rev{and} the posterior truly mean.
In plain language, the model likelihood is the probability of the data given a model.
Without a model, there can be no calculation of the model  likelihood.

Priors can be set on parameters in the model, specifying distributions from which the value is thought to be drawn.
These distributions are often based on the researcher's intuition, and on information from prior studies.
The posterior distribution is a set of plausible solutions given the model likelihood and the prior.
During Bayesian estimation, different values will be sampled for model parameters.
Their probability will be evaluated according to the likelihood and the prior.
Therefore, the posterior is proportional to the likelihood and the prior.
A good solution will often appear in the posterior sample many times.

In phylogenetics, we often refer to our models as continuous-time Markov chains.
`Continuous-time' refers to models allowing change between character states to occur instantaneously at any point in an evolutionary history.
Changes in the character state \rev{are} not confined to the node; instead, branch lengths on a phylogeny are proportional to the number of expected changes per character along that particular branch. 
%In this context, `Markov chain' refers to the joint probability distribution including all the parameters for the model of morphological substitution, the model of molecular substitution, and the tree and clock models. %Joelle: the Markov chain component refers to the fact that these models are memory-less, i.e. that the event probabilities at any point in the process depend only on the current state and not on the history of the process - so it's not an inherent property of phylogenetic models, just the existing ones
\revJ{In this context, `Markov chain' refers to the fact that (existing) phylogenetic models are memory-less. This means the joint probability --- taking into account all the parameters for the model of morphological substitution, the model of molecular substitution, and the tree and clock models --- depends only on the current state and not on the history of the process}.
In practice, this is the computer model that we use to estimate the posterior \citep{Hoehna2016b}.
\end{boxedtext}

\begin{boxedtext}{Hierarchical Models}
The tripartite approach to divergence time estimation is what is termed a \textit{hierarchical model}. 
Hierarchical models are models in which variation may be described by different submodels.
In the case of divergence time estimation, the character data \rev{(molecular and/or morphological)} is described by one model, such as the Mk model.
The distribution of evolutionary rates across branches is described by the clock model.
Finally, the distribution of speciation, extinction and fossil sampling is described by the tree model.
Together, these three components are used to estimate a tree, branch lengths in units of time, and other relevant model parameters.

This term may be confusing, as model components may have a hierarchy of priors. 
For example, if we placed a lognormal distribution with shape parameter 10 on the mean clock rate, this is a prior.
If instead, we placed an exponential prior on the shape parameter \rev{of the lognormal distribution}, that exponential prior is called \textit{hyperprior}.
This, while a hierarchy of priors, is not a hierarchical model in the same way that the complete tripartite model for divergence time estimation is hierarchical.

See \citet{Heath2012a} for a nice example of the hyperprior approach to modeling uncertainty in the parameters associated with fossil calibration densities.
\end{boxedtext}

\begin{boxedtext}{Maximum Likelihood and Bayesian Estimation}
As discussed in the section `The Likelihood', the probability of the data is calculated given a model. 
In maximum likelihood estimation, models are proposed, and the likelihood of the data is calculated given each of those models.
The model that gives the \rev{highest} likelihood is considered to be `the best'.
This is generally a point estimate returning one tree, one set of branch lengths, and one set of other model parameters.
See \citet{Holder2003} and \citet{Yang2012} for more information on the details and history of these different approaches.

Inference of undated trees from \rev{molecular and/or} morphological data can be accomplished in many pieces of maximum likelihood software, such as PAUP \citep{Swofford2003}, RAxML \citep{Stamatakis2014}, IQTREE \citep{IQtree}, and GARLI \citep{zwickl2006}.
Estimation of dated trees incorporating \rev{molecular and/or} morphological data has mostly been accomplished in a Bayesian context, using software such as MrBayes \citep{Huelsenbeck2002, Ronquist2003}, BEAST \citep{BEAST}, BEAST2 \citep{BEAST2}, MCMCTree \citep{MCMCtree}, and RevBayes \citep{Hoehna2014b, Hoehna2016b}.
While there is no reason models such as the FBD cannot be estimated using maximum likelihood, in practice, it is not straightforward to incorporate the uncertainty associated with parameters within a maximum likelihood framework.
\end{boxedtext}
