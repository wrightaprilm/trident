\caption{\textbf{Undated versus dated phylogenetic inference.}
A phylogenetic tree is comprised of \textit{tips} (indicated with circles in panel A), which represent the taxa between which we aim to infer the evolutionary relationships.
These taxa are connected by \textit{branches}. 
The branches are connected by \textit{nodes} (indicated with triangles in panel A), which reflect the most recent common ancestor between two given tips. 
The overall structure of the tree used to represent phylogenetic relationships is referred to as the \textit{topology}.
In an undated phylogeny branch lengths are typically in units that represent the overall amount of character change, \rev{indicated here by the scale bar. In undated model based tree inference, the units usually represent the number of expected changes per character}.
A tree estimated with no temporal information can be seen in panel A.
In a time calibrated tree the branch lengths will be in units of calendar time, often in years or millions of years. 
Panel B shows the same tree from panel A, but with branches in millions of years, along with stratigraphic ranges, i.e., the interval between first and last appearance times (grey boxes).}
\label{fig:undated}