\caption{\textbf{A tripartite model for Bayesian divergence time estimation.}
The top panel shows the key ingredients required during inference.
The data used to generate time calibrated trees: molecular or morphological phylogenetic characters, and age information, typically fossil sampling times.
The model includes the substitution (site) model, which describes the evolution of characters, the clock model, which describes the distribution of evolutionary rates across the tree, and the tree model, which describes the distribution of speciation events across the tree.
Bayes theorem is presented in the middle panel. 
The bottom panel illustrates how everything comes together for the Bayesian estimation of divergence times.
This figure is based on Fig. 1 in \citet{duPlessis2015}.
\revJ{Note this formulation of Bayes' theorem treats fossil ages as fixed. In reality, fossils are associated with a range of ages and this uncertainty should be reflected through the use of priors. 
See \citet{Drummond2016} and \citet{BaridoSottani2019a} for details.}}
\label{fig:bayes}