\caption{\textbf{Phylogenetic $Q$ matrices.}
In this schematic, we have several representations of different types of character change.
For nucleotide data, we know that we are more likely to see certain types of change, such as two-ringed bases (purines) transitioning to other two-ringed bases, and one-ringed bases transitioning to other one-ringed bases.
This is represented by thicker arrows connecting these bases.
On the other hand, for morphological data, character states do not carry common meaning across characters.
At one character, changing, for example, from a `0' state to a `1' state may be a small change.
At another, it may mean gaining a complex character.
Therefore, researchers have largely used the Mk model of Lewis (2001) to model these data.
The schematic below shows the assumption of equal change probability between states.}
\label{fig:Q}