\caption{\textbf{A schematic showing different clock models}, and what they mean for the distribution of evolutionary rates across the tree. \rev{For each clock type, a set of sample distributions are shown. These distributions demonstrate how the distribution will look if a different prior is selected for its underlying parameters. An arrow indicates which distribution was used to simulate the rates shown on the sample tree.} Row one shows an uncorrelated clock, with branch rates drawn from the exponential distribution. Because this clock is uncorrelated, a descendent may have a very different rate of evolution than its ancestor.  In the second row, an autocorrelated clock, rates of evolution in the ancestor and descendant are expected to be more similar. \rev{As can be seen in the set of sample distributions, low values for the exponential rate parameter or the lognormal log variance parameter result in very wide distributions, implying that there can be a wide range of evolutionary rates across the tree. When the rate or log variance parameters are high, the rates are more constrained.} The third row shows Dirichlet-distributed rates. This is a biologically agnostic clustering method for assigning branch rates. \rev{As can be seen in the distributions for this parameter, a high shape parameter implies a strong central tendency, and low values imply more variation in rates. \rev{Code to reproduce this figure is provided in the supplement.}}}
\label{fig:distn}