\section{\sout{Incorporating time}}

\rw{Most of the appears in or has been moved to section 3}

\sout{The site model does not tell us anything about absolute time.
When estimating a tree from character data using a site model, the estimated phylogeny is not scaled to time.
As discussed above, a branch length on a tree in a Bayesian model is the number of expected substitutions per site.
This is a rate: the number of substitutions multiplied by the units of time.
It is an expected rate, because Bayesian and likelihood methods allow for hidden state changes.
Therefore, the number of expected changes could be larger than how many we observe in the data we collected.
Ultimately, we want to be able to disentangle this rate and time in order to estimate time since divergence on a tree.
We need to convert the rate into \textit{expected substitutions per site per calendar unit time}.}


\sout{To tease apart rate and time, we either need to know the average substitution rate (this is rarely known outside \edit{the context} of DNA characters for some model organisms, see early demonstrations of difficulties applying a global clock in \cite{gaut1992, MOOERS1994, bromham1996, rambaut1998}) or we need to calibrate the substitution rate using temporal information from elsewhere.
Temporal information typically comes from  fossil sampling times.
Alternatively, this info can come from biogeographic data, or from human-recorded data in the case of phylogenetic problems concerning recent timescales.
This information is then modeled using hierarchical  Bayesian models to estimate time since divergence.
The model is hierarchical because it links together different sub-models (i.e. the substitution, clock and tree models) (see box).
There are two key model components required to date a phylogeny: the clock model and the tree model.
Rate and time are often only semi-identifiable.
A model is \textit{identifiable} when different values for the parameters generate different probabilities for the observed data.
On the converse, a model is not identifiable when multiple sets of parameters could generate the same probabilities of the observed data.
In this case, we may be unable to identify, or distinguish, the true parameter values.
In practice, this means we need to put strong prior information on the average substitution rate and times. 
A consequence of this that the results will be very sensitive to these prior choices.
As a result, it is important for biologists to understand each of the component pieces (the clock and tree models) in order to make good choices in selecting and parameterizing a prior.
We will begin with the clock model.}





\begin{boxedtext}{Summarizing the posterior sample of trees}

\rw{Do we need this info after all?}

Bayesian methods return a sample of phylogenetic trees that are plausible under the model of evolution, and the priors chosen by the researcher. 
This is a nice property: phylogenetic estimation is known to be prone to error when the assumed model of evolution does not match the process of evolution that generated the data.
Because we cannot truly know all the evolutionary forces that generated our observed data, it is wise to take into account uncertainty about the tree and branch lengths.

However, most researchers do still have a need to summarize and display trees. 
Here, we discuss a few common tree summarization strategies.  

\textbf{Maximum clade credibility tree:} In this summary, each tree in the posterior sample is evaluated.
Each clade on the tree is scored for how often that clade is seen in the posterior sample.
The tree then receives a score that is the product of the scores of all the clade scores on the tree.
The tree with the highest score is reported as the summary. 

\textbf{Maximum \textit{a posteriori} tree:} This is the tree with the greatest posterior probability in the sample.
This value does not only account for relationships. 
The posterior will average over branch lengths and other facets of the model, as well.

\textbf{Consensus tree:} This tree displays the most common relationships found in the posterior sample. 
A strict consensus tree displays only the clades found in every tree.
These are not commonly used as they result in little to no resolution in the tree.
Semi-strict consensus trees, such as majority-rule, in which clades that appear in more than 50\% of the trees in the sample are included in the summary tree, are more common. 
This type of summary tree can result in a topology that was not actually sampled in the analysis, but on which all clades are well-supported. 
Because parsimony remains well-used in morphological phylogenetics, and parsimony trees are summarized by consensus, this family of methods is common in the literature.



\end{boxedtext}