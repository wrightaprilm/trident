
Because the prior distribution can affect the parameters estimated, Bayesian methods have a suite of well-developed statistical approaches for evaluating the fit of both the model and the priors to the data.
Called Bayes Factors \cite{Xie2011}, these metrics describe the support for one model, and all its associated priors, over another model. 
These methods weigh the posterior evidence of two models against one another. 
These methods can be used to compare non-nested or mixture models.
It is worth noting, however, that the Bayes Factor can only provide evidence in favor of one model.
It cannot tell a researcher if the model is adequate; that is, capturing important facets of the process of evolution.
Other methods are available that can be used to assess model adequacy \cite{Brown2009, Brown2014}.